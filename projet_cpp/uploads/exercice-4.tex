\documentclass[10pt,a4paper]{article}
\setlength{\oddsidemargin}{0pt}
\setlength{\evensidemargin}{0pt}
\setlength{\headsep}{25pt} % Entre le haut de page et le texte
\usepackage{setspace}
\usepackage{geometry}
\geometry{hmargin=1.8cm,vmargin=3cm}
\usepackage{graphicx}
\usepackage{setspace}
\usepackage{fancybox}
\usepackage{fancyhdr}
\usepackage [latin1]{inputenc}
\pagestyle{fancy}
\renewcommand{\headrulewidth}{0pt}
\singlespacing
\usepackage[french]{babel}
\everymath{\displaystyle}
\usepackage{amsmath,amsfonts,amssymb}
\usepackage{amsthm}
\usepackage{wasysym}
\usepackage{tikz}
\usepackage{pgfplots}
\pgfplotsset{compat=1.15}
\usepackage{mathrsfs}
\usetikzlibrary{arrows}
\usepackage{tkz-tab}
\usepackage[tikz]{bclogo}
\usepackage{tcolorbox}
\usepackage{shadethm}
\usepackage{thmtools}
%\usepackage[S,cut=false, underline=false]{thmbox}
\usepackage[M,cut=false, underline=false, thickness=0pt]{thmbox}


\newcounter{nexercice}
\setcounter{nexercice}{3}

\newenvironment{exercice}{
\refstepcounter{nexercice}
\begin{small}\noindent \Ovalbox{\textbf{\textsf{Exercice } \arabic{nexercice} :}}\hskip 2mm}{\end{small}\vskip 0.5cm}


\begin{document}
\begin{exercice}
\begin{enumerate}
\item Soit $x\in \mathbb{R}_+,\ \forall n\in \mathbb{N}^*,\ n^{3/2}\dfrac{x}{n^2+x^2} \underset{n\to +\infty}{\sim} \dfrac{x}{\sqrt{n}} \underset{n\to +\infty}{\longrightarrow} 0$. Donc par crit�re de Riemann, $\sum_{n\geq 1} u_n(x)$ converge.\\ Ainsi $\sum_{n\geq 1} u_n$ converge simplement sur $\mathbb{R}_+$.
\item Soit $n\in \mathbb{N}^*,\ \forall x\in \mathbb{R}_+,\ u^{\prime}_{n}(x)=\dfrac{n^2+x^2-2x^2}{{(n^2+x^2)}^2}=\dfrac{n^2-x^2}{{(n^2+x^2)}^2}$.\\ \\On a les variations suivantes :
\begin{flushleft}
\begin{tikzpicture}
\tkzTab[lgt=4,espcl=4] %Largeurs des deux colonnes
{ $x$  / 1,  Signe de $u^{\prime}_{n}(x)$ / 1, Variations de $u_n$ / 2} %Titres des lignes / sur combien de lignes les �crire
{ $0$ , $n$ ,$+\infty$} %Valeurs premi�re ligne
{ ,+,z,-,} %Signes + valeurs num�riques et z donne 0 avec des pointill�s
{ -/$0$ , +/$\dfrac{1}{2n}$ , -/ $0$} %Variation / valeur
\end{tikzpicture}
\end{flushleft}
Donc $\underset{x\in \mathbb{R}_+}{\sup} \lvert u_n(x)\rvert =\dfrac{1}{2n}$.\\
Comme la s�rie $\sum \dfrac{1}{2n}$ est une s�rie de Riemann divergente, $\sum u_n$ ne converge pas normalement sur $\mathbb{R}_+$.
\item Soit $a>0$, on a\\
$\forall n\in \mathbb{N}^*,\ \forall x\in [0,a],\ \lvert u_n(x)\rvert = \dfrac{x}{n^2+x^2} \leq \dfrac{a}{n^2}$.\\
Puisque $\sum \dfrac{a}{n^2}$ est une s�rie de Riemann convergente ($2>1$), on en d�duit que $\sum u_n$ converge normalement sur $[0,a]$.
\end{enumerate}
\end{exercice}
\end{document}
