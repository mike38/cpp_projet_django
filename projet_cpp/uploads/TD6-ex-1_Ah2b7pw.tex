\documentclass[a4paper,11pt]{article}
\usepackage[latin1]{inputenc}
\usepackage[frenchb]{babel}
\usepackage{array,multicol,enumerate}
\usepackage{amssymb,amsmath,amsfonts,amsthm,geometry}
\newenvironment{questions}{\begin{enumerate}}{\end{enumerate}}

\begin{document}
On dispose d'une infinit\'e d'urnes num\'erot\'ees, de telle sorte que l'urne num\'ero k soit constitu\'ee de $2^k$ boules dont une seule blanche ($k \in \mathbb{N}^*$). On suppose de plus que la variable al\'eatoire
\'egale au num\'ero de l'urne choisie, suit une loi g\'eom\'etrique de raison $\dfrac{1}{2}$.\smallskip\\
On choisit une urne au hasard puis on tire une boule dans cette urne. D\'eterminer la probabilit\'e d'obtenir une boule blanche.
\end{document}

